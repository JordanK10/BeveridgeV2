My strategy here is to first derive equations for an individual firm to maximize its profits and write down a micro-founded ABM for unemployment and vacancies at the level of individual firms (as we will see, the resulting model will be very simple).  Then we can derive an aggregate equation for unemployment and vacanciesfor the case with two firms.  We will assume that firms would like to adjust their employees to maximize profits.  They are prevented from doing this by frictions, but they will still maximize profits as best they can.

To maximize profits firms to adjust production to match demand.  Assume all firms have a linear production function so that production is proportional to the number of employees.  Also assume that unemployed workers will take the best job available, and the matching function $m$ accounts for any frictions in finding a job.    Assume for now that all firms have the same matching function that depends on aggregate unemployment $u$ and vacancies $v$.  To make the dependence of matching on vacancies explicit, I am rewriting the matching function as $m(u,v) = m'(u,v)/v$, where $m'$ is the conventional form of the matching function.  In this case the matching rate for a given firm $i$ that posts vacancies $v_i$ is $v_i m(u,v)$.  Thus the simple matching function $K uv$ in the purely aggregate context is replaced here by $K u v_i$, with $m = K u$.  This means that firms that place more vacancies will hire workers faster but they also have more workers to hire. 

Vacancies and unemployment are fundamentally different because vacancies can be adjusted by the firm up or down at will, whereas the firm can lower its employment by firing people but it can't raise it by un-firing them -- it has to wait for vacancies to be filled.  Employment increases when jobs are filled and decreases when workers are fired or when they spontaneously separate.  Vacancies change because the firm adjusts vacancies upward by posting new ones or downward by withdrawing existing postings.  However, the firm cannot have less than zero vacancies, which we will see is very important. 

During an upturn a firm will increase its vacancies to hire more workers.  During a downturn, a firm has a choice between reducing its vacancies or firing workers or some combination of the two.  We have to make an assumption about how firms will make this choice.   A plausible behavioral assumption is that firms will stop hiring (setting vacancies to zero) before they start firing.  There are lots of reasons why a firm would prefer to stop hiring before firing, such as being kind to their employees, an unwillingess to create bad workforce morale or procedural simplicity.  Following this policy generates highly nonlinear behavior:  During a small downturn, the firm will reduce its vacancies as needed, without firing any workers.  However, as expected demand drops, once the firm's vacancies reach zero, it will fire workers as needed to match expected demand.   Both of these actions (firing and reducing vacancies) can happen instantaneously, in contrast to hiring workers, which takes time. Thus, during an upturn it will take time for a firm to hire workers and it will not be able to maximize profits perfectly, whereas during a (sufficiently large) downturn the firm has the levers it needs to satisfy Equation (2) and maximize profits.

Let $D_i (t)$ be the expected demand for firm $i$ at time $t$.  Under the assumptions above, this translates into an employee target $\hat{e_i} (t) = D_i/\pi_i$, where $\pi_i$ is the average productivity per worker of firm $i$.  This means that the targeted change in number of employees for the firm is $D_i/\pi_i - e_i$.  For convenience, for the moment assume that $\pi_i$ is constant in time, and assume that the firm always expects to lose a fraction $s$ of its employees per unit time as a result of spontaneous separations, e.g. retirement.

The dynamics of vacancies and unemployment can be written
\begin{eqnarray}
v_i (t) & = & \left[ D_i/\pi_i - e_i + e_i s/m \right]^+ \label{vacancy}\\
\frac{de_i}{dt} & = & v_i m  - e_i s + \left[ D_i/\pi_i - e_i \right]^-\label{employmentchange},
\end{eqnarray}
where $[x]^+ = x$ if $x > 0$ and equals zero otherwise, and vice versa for $[x]^-$. 


\textbf{Note from Jordan: in Eq. 2, the minus sign from in from of the third term needs to be plus, because the term in the bracket is always non-positive.}

Assume that demand for each type of firm is given exogenously.  We expect that the sensitivity of the expected demand for the output of different firms to vary widely.  At one extreme there are firms for which it will be effectively constant, so that vacancies and the number of employees are constant, i.e.
\begin{eqnarray}
v_i (t) & = & e_i s/m\\
\frac{de_i}{dt} & = & 0.
\end{eqnarray}
At the other extreme will be firms whose demand varies widely.  Note that previously we had a separate parameter for $v_0$; the equilibrium condition connects this to $s$, so that it is no longer needed.  Also note that we no longer have a separate differential equation for vacancies, which are now just a function of the gap between current employment and expected demand.

In order to complete the model we need to compute the aggregate unemployment and vacancies, which are required as inputs for the matching function.  Letting $L$ be the size of the labor force, I believe this is:
\begin{eqnarray}
u  & = & 1 - e = 1 - \frac{\sum_{i=1}^N e_{i}}{L}\\
v & = &  \frac{\sum_{i=1}^N v_{i}}{\sum_{i=1}^N v_{i} + \sum_{i=1}^N e_{i}}
\end{eqnarray}
I am not sure about how vacancies are normally defined -- Maria, is this correct?  Here for simplicity I will assume that vacancies are also defined relative to the size of the labor force, i.e.
\begin{equation}
v =  \frac{\sum_{i=1}^N v_{i}}{L}.
\end{equation}
To summarize, the model consists of a set of $N$ differential equations for the employment of each firm, which are weakly coupled through their common dependence on the matching function $m$.

 In general the demand for each good might be unrelated, but we expect that they will all be influenced by a common factor, like GDP, so that their expected demands are cross sectionally correlated. Suppose, for example, that the demand for firm $i$ is 
\begin{equation}
D_i (t)/\pi_i = \sigma_i (1 + C_i G(t)),
\end{equation}

(if L is set to the maximum of D/pi, then the unemployment should go to zero around the time of maximum demand (G(t) is max). Sensibly, set L to be larger. When the economy is at the point of full output, what percent of the labor pool is employed? Sigma should be described relative to the population size)

where $\sigma_i$ can be thought of as the "size" of the firm and $G(t)$ is a common signal about the state of the economy (which might or might not be related to GDP) that firm's use to evaluate expected demand  and $C_i$ is a coefficient that measures the sensitivity of the firm's demand to the state of the economy.  Since firm's only differ in their sizes and sensitivities, we could lump together all firms with similar sensitivity.