If the firm size is given by $\sigma_i$, the unemployed population is given by $U=L-\sum_{i=1}^Ne_i$, then we can set $\sigma_i$ to something sensible given the size of the economy. Consider the constraint
\begin{equation}
    \sigma_i=L(k_i/N-\xi).
\end{equation}
If firm weight $k_i=1$ for all firms, then firms are uniform in size. We could also assign 
\begin{equation}
    k_i=2i/(N+1)
\end{equation} for an ascending firm size or impose a power law firm size relationship such that $k_i=i^{-a}A$ where $a$ with normalization $A$. Generally, we constraint $k_i$ such that $\sum_ik_i=N$. The surplus parameter $\xi$ sets the size of the unemployment pool when all firms are at occupational capacity. For example, if we desire a surplus of 5\%, we assign $\xi=.05$. 

Now, what's left is to constrain the population size. 
We can constrain with the data directly which will be useful for calibrating the data when real data max and capacity are known. 
Assume that the population is set to the maximum production target, ensuring a relationship between the labor pool and th potential productivity of that pool in the data.  Assuming $t_\textrm{max}$ is the timestep at which this condition is satisfied, the population would look something like $L=[D(t_{max})/\pi]+\xi$ when $N=1$. More generally, this could be $L=\sum_{i=1}^N[D_i(t_{\textrm{max}})/\pi_i]+\xi$.
