Consider an aggregate equilibrium industry size size $\sigma$ and firm share $k$.
The size of firm $i$ is given by $\sigma_i=k_i\sigma$
If firm weight $k_i=1$ for all firms, then firms are uniform in size. We could also assign 
\begin{equation}
    k_i=2i/(N+1)
\end{equation} for an ascending firm size or impose a power law firm size relationship such that $k_i=i^{-a}A$ where $a$ with normalization $A$. Generally, by constraining $k_i$ such that $\sum_ik_i=N$, we guarantee that $\sum_i\sigma_i=\sigma$.
 
Using the aggregate firm size and information about the data, we can constrain the population size.
This esnures that as employment maximizes given the signal, unemployment rate converges to some value $\xi$.
This method will be helpful for constraining the model in terms of extrema in the data.
Assuming $t_\textrm{max}$ is the timestep where demand is maximum, the population would look something like 
\begin{equation}
L=\Big[\sum_{i=1}^N[D_i(t_{\textrm{max}})/\pi_i\Big](1+\xi)
\end{equation}
For a single firm, this is $L=\sigma(1+CG_\textrm{max})(1+\xi)$.